\documentclass{article}
\usepackage{emaxima}
\usepackage{url}

\newcommand{\emx}{\textsl{\sffamily EMaxima}}
\newcommand{\mx}{\textsl{\sffamily Maxima}}
\newdimen\firstcol
\firstcol=.35\textwidth
\newdimen\secondcol
\secondcol=.65\textwidth

\title{A Quick and Dirty Guide to \emx{}}
\author{Jay Belanger}
\date{}

\begin{document}

\maketitle

\section{Introduction to \emx}

\emx{} is a major mode for Emacs that allows the user to write documents
while interacting with \mx.  It is based on Dan Dill's
\TeX{}/\textit{Mathematica} package\footnote{\TeX/\textit{Mathematica} is
available from \url{ftp://chem.bu.edu/pub/tex-mathematica-2.0}.}, and
uses a modified version of William Schelter's
\texttt{maxima.el}.
While the \texttt{maxima-mode} provided by \texttt{maxima.el} is designed
to help write \mx{} programs, \emx{} is designed to help write
documents that include \mx{} code.  \emx{} is an extension of the
\LaTeX{} mode provided by AUC\TeX{}\footnote{This can be configured so
that \emx{} extends the standard \TeX{} mode provided by Emacs, or just
text mode.}, and so has the \LaTeX{} mode commands available.  The
resulting document can be processed by \LaTeX{}; this requires putting 
\begin{verbatim}
\usepackage{emaxima}
\end{verbatim}
\noindent
in the preamble.

\section{Cells}

The basic unit of \mx{} code in \emx{} is a \textbf{cell}.  A cell
consists of text between the delimiters
\begin{verbatim}
\beginmaxima
\end{verbatim}
\noindent
and
\begin{verbatim}
\endmaxima
\end{verbatim}
\noindent
A cell can be created by typing \texttt{C-c C-o}.  (The \texttt{C-o} in this
case stands for \textbf{o}pening a cell.)  The delimiters will then be
placed in the buffer, and the point will be placed between them.

When working with several cells, you can jump between them by using
\texttt{C-c +} to go to the next cell and \texttt{C-c -} to go to the
previous cell.

\section{Evaluating cells}

\noindent
To evaluate the contents of a cell, the command
\texttt{C-cC-uc} (\texttt{emaxima-update-cell})\footnote{Sending the
  cells contents to a \mx{} process and returning the results is
  called \textbf{updating} the cell, the prefix 
\texttt{C-c C-u} will be used to update cells in different ways.} 
will send the contents
of the cell to a \mx{} process (if there is no \mx{} process running,
one will be started) and return the results to the cell,
separated from the input by the marker
\begin{verbatim}
\maximaoutput
\end{verbatim}
\noindent
To differentiate
$sin(x^2)$, for example, type 
\texttt{diff(sin(x\^{}2),x);} in a cell:
\begin{verbatim}
\beginmaxima
diff(sin(x^2),x);
\endmaxima
\end{verbatim}
\noindent
After typing \texttt{C-c C-u c}, it will look like
\begin{verbatim}
\beginmaxima
diff(sin(x^2),x);
\maximaoutput
                                           2
                                  2 x COS(x )
\endmaxima
\end{verbatim}
\noindent
To delete the output and return the cell to its original form, you can
use the command \texttt{C-c C-d}.
If the document is to be \TeX{}ed, the above cell will look like:
\beginmaxima
diff(sin(x^2),x);
\endmaxima
and the cell with output will look like:
\beginmaxima
diff(sin(x^2),x);
\maximaoutput
                                           2
                                  2 x COS(x )
\endmaxima

\emx{} mode can take advantage of the fact that \mx{} can give its
output in \LaTeX{} form.  The command \texttt{C-c C-u C}
works the same as \texttt{C-c C-u c}, except now the output is in \LaTeX{}
form, ready to be formatted by \LaTeX{}.  In general, if 
\texttt{C-c C-u }\textsl{letter} returns \mx{} output, then
\texttt{C-c C-u }\textsl{capital letter} will return the output in
\TeX{} form.  The above cell would become
\begin{verbatim}
\beginmaxima
diff(sin(x^2),x);
\maximatexoutput
$$   2\*x\*\cos x^{2} $$
\endmaxima
\end{verbatim}
\noindent
which, when \LaTeX{}ed, would become
\beginmaxima
diff(sin(x^2),x);
\maximatexoutput
$$   2\*x\*\cos x^{2} $$
\endmaxima
\noindent
(Note that whenever a cell is updated, any old output is discarded and
replaced with new output.)  The command \texttt{C-c C-u a} will update all
of the cells in your document, 
stopping at each one to ask if you indeed want it updated.  Given an
argument, \texttt{C-u C-c C-u a}, it will update all of the cells in your
document without asking.  The command \texttt{C-c C-u A} behaves
similarly, except now all the output is returned in \LaTeX{}  form.

\section{Initialization Cells}

\noindent
It is possible that you want certain cells evaluated separate from the
others; perhaps, for example, you want certain cells evaluated whenever
you open the document.  This can be done using initialization cells.
An initialization cell is delimited by
\begin{verbatim}
\beginmaxima[* Initialization Cell *]
\end{verbatim}
\noindent
and
\begin{verbatim}
\endmaxima
\end{verbatim}
\noindent
The command \texttt{C-c C-t} will turn a cell into
an initialization cell, applying \texttt{C-c C-t} again will turn it
back into a regular cell.  
When \LaTeX{}ed, an initialization cell will look like
\beginmaxima[* Initialization Cell *]
diff(sin(x^2),x);
\endmaxima

Initialization cells behave like regular
cells, except that they can be treated as a group.
To evaluate all initialization cells (without displaying the output in
the document buffer), the
command \texttt{C-c C-u t} will go to each of the
initialization cells and evaluate them.
If you want the output of the initialization cells to be brought back 
to the document buffer,  stopping at each one to see it
you indeed want it updated, then use the command \texttt{C-c C-u i}.
With an argument, \texttt{C-u C-c C-u i}, the
initialization cells will be updated without asking.   The command 
\texttt{C-c C-u I} behaves just like \texttt{C-c C-u i},
except that the output is returned in \TeX{} form.

\section{Referencing Other Cells}

\noindent
Instead of \mx{} code, a cell can contain a reference to another cell,
and when the original cell is sent to \mx{}, the reference is replaced
by the referenced cell's contents (but only in the \mx{} process
buffer, the cell's 
content in the document's buffer is not changed).  In order to do
this, the original cell must be marked by having a label of the form
\texttt{<}\textsl{filename}\texttt{:}\textsl{cell label}\texttt{>}.
(The reason for the \textsl{filename} will become apparent later, and
\textsl{cell label} is optional for the referencing cell.)
The referenced cell must also be labeled, with the same
\textsl{filename} but a unique \textsl{cell label}.  To reference the
other cell, the original cell need only contain the marker for the
referenced cell.  For example, given cell 1:
\begin{verbatim}
\beginmaxima<filename:optional>
<filename:definef>
diff(f(x),x);
\endmaxima
\end{verbatim}
\noindent
and cell 2:
\begin{verbatim}
\beginmaxima<filename:definef>
f(x):=sin(x^2);
\endmaxima
\end{verbatim}
\noindent
then the result of updating cell 1 (\texttt{C-c C-u c}) will be:
\begin{verbatim}
\beginmaxima<filename:optional>
<filename:definef>
diff(f(x),x);
\maximaoutput
                                             2
                                f(x) := SIN(x )
                                           2
                                  2 x COS(x )
\endmaxima
\end{verbatim}
\noindent
When \LaTeX{}ed, the top line will contain a copy of the marker.

\newpage

\beginmaxima<filename:optional>
<filename:definef>
diff(f(x),x);
\maximaoutput
                                             2
                                f(x) := SIN(x )
                                           2
                                  2 x COS(x )
\endmaxima

A cell can contain more than one reference, and referenced cells can
themselves contain references.  

To aid in labelling the cells, the command \texttt{C-c C-x}
will prompt for a label name and label the
cell.  To aid in calling references, the command \texttt{C-c C-TAB}
can be used for completing the
the \textsl{filename} and \textsl{cell label} parts of a reference, 
based on the current labels.  
Another option is to set the Emacs variable
\texttt{emaxima-abbreviations-allowed} to \texttt{t}, say, by putting
the line
\begin{verbatim}
(setq emaxima-abbreviations-allowed t)
\end{verbatim}
\noindent
in your \texttt{.emacs} file.  This will allow the \textsl{filename}
and \textsl{cell label} parts of a reference to be abbreviated by enough
of a prefix to uniquely identify it, followed by ellipses
\texttt{...}
For example, if there are cells labelled
\begin{verbatim}
<filename:long description>
<filename:lengthy description>
\end{verbatim}
\noindent
Then the reference
\begin{verbatim}
<...:le...>
\end{verbatim}
\noindent
will suffice to refer to the second label above.

If you want the references in a cell to be replaced by the actual
code, the command \texttt{C-c @} will expand all the
references and put the code into a separate buffer (so it will not
affect the original document).

\section{WEB}

\noindent
The reason for the ability to reference other cells is so that you can
write what Donald Knuth calls literate programs.  The idea is that the
program is written in a form natural to the author rather than natural
to the computer.  (Another aspect of Knuth's system is that the code
is carefully documented, hence the name ``literate programming'', but
that is done naturally in \emx{}.)  Knuth called his original
literate programming tool \texttt{WEB}, since, as he puts it,
``the structure of a software program may be thought of as a web that
is made up of many interconnected pieces.''  
\emx{}'s ability in this respect is taken directly from
\TeX{}/\textit{Mathematica}, and is ultimately based on
\texttt{WEB}. To create a 
program, the ``base cell'' or ``package cell'' should contain 
a label of the form \texttt{<}\textsl{filename}\texttt{:>} 
(no cell label), and can
contain references of the form 
\texttt{<}\textsl{filename}\texttt{:}\textsl{part}\texttt{>}
(same file name as the base cell).  

As a simple (and rather silly) example, suppose we want to create a
program to sum the first $n$ squares.  We could start:
\begin{verbatim}
\beginmaxima<squaresum.max:>
squaresum(n) := (
  <squaresum.max:makelist>
  <squaresum.max:squarelist>
  <squaresum.max:addlist>
  );        
\endmaxima
\end{verbatim}
\noindent
We would then need cells
\begin{verbatim}
\beginmaxima<squaresum.max:makelist>,
L:makelist(k,k,1,n),
\endmaxima

\beginmaxima<squaresum.max:squarelist>
<squaresum.max:definesquare>
L:map(square,L),
\endmaxima

\beginmaxima<squaresum.max:addlist>
lsum(k,k,L)
\endmaxima
\end{verbatim}
\noindent
and then we would also need:
\begin{verbatim}
\beginmaxima<squaresum.max:definesquare>
square(k) := k^2,
\endmaxima
\end{verbatim}
\noindent
When \TeX{}ed, the header of the cell will say that it determines the
file \texttt{squaresum.mu}.  
\beginmaxima<squaresum.max:>
squaresum(n) := (
  <squaresum.max:makelist>
  <squaresum.max:squarelist>
  <squaresum.max:addlist>
  );        
\endmaxima

The command 
\texttt{C-u C-c @} will put all the pieces
together in the file it determines.  The resulting file, in this case,
will be \texttt{squaresum.max} and will look like:
\begin{verbatim}
squaresum(n) := (
  L:makelist(k,k,1,n),
  square(k) := k^2,
  L:map(square,L),
  lsum(k,k,L)
  );        
\end{verbatim}
\noindent
(Although the idea is that only the computer need look at this file.)

\section{Other types of cells}

\noindent
When a cell is \TeX{}ed, the input and output are kept separate.  To
have the results look like a \mx{} session, there are, in addition to
the standard cells, special cells called \emph{session cells}.   A
session cell is delimited by
\begin{verbatim}
\beginmaximasession
\end{verbatim}
\noindent
and
\begin{verbatim}
\endmaximasession
\end{verbatim}
\noindent
The command \texttt{C-c C-p} will create a session cell.  When a
session cell is updated, the output will be marked off with
\verb+\maximasession+, and will contain both the input and the output,
with the \mx{} prompts.  For example, if the session cell
\begin{verbatim}
\beginmaximasession
diff(sin(x),x);
int(cos(x),x);
\endmaximasession
\end{verbatim}
\noindent
were updated, the result would look like
\begin{verbatim}
\beginmaximasession
diff(sin(x),x);
integrate(cos(x),x);
\maximasession
(C1)diff(sin(x),x);

(D1)                                COS(x)
(C2)integrate(cos(x),x);

(D2)                                SIN(x)
\endmaximasession
\end{verbatim}
\noindent
which, when \TeX{}ed, would look like
\beginmaximasession
diff(sin(x),x);
integrate(cos(x),x);
\maximasession
(C1)diff(sin(x),x);

(D1)                                COS(x)
(C2)integrate(cos(x),x);

(D2)                                SIN(x)
\endmaximasession
\noindent
If it is updated in \TeX{} form, it will look like
\begin{verbatim}
\beginmaximasession
diff(sin(x),x);
integrate(cos(x),x);
\maximatexsession
\C1.  diff(sin(x),x); \\
\D1.   \cos x \\
\C2.  integrate(cos(x),x); \\
\D2.   \sin x \\
\endmaximasession
\end{verbatim}
\noindent
which, when \TeX{}ed, will look like
\beginmaximasession
diff(sin(x),x);
integrate(cos(x),x);
\maximatexsession
\C1.  diff(sin(x),x); \\
\D1.   \cos x \\
\C2.  integrate(cos(x),x); \\
\D2.   \sin x \\
\endmaximasession

For particularly long output lines inside the \verb+\maximatexsession+
part of a session cell, the command \verb+\DD+ will typeset anything
between the command and \verb+\\+.  Unfortunately, to take advantage
of this, the output has to be broken up by hand.
If a session cell has not been updated, or has no output for some
other reason, it will not appear when the document is \TeX{}ed.

There is one other type of cell, a \emph{noshow cell}, which can be
used to send \mx{} a command, but won't appear in the \TeX{}ed
output. A noshow cell can be created with \texttt{C-c C-n}, and will
be delimited by
\begin{verbatim}
\beginmaximanoshow
\end{verbatim}
\noindent
and
\begin{verbatim}
\endmaximanoshow
\end{verbatim}

Session cells and noshow cells cannot be initialization cells or part of
packages.\footnote{That could be changed, but I don't know why it'd be
useful.} 

If the command to create one type of cell is called while inside
another type of cell, the type of cell will be changed.  So, for
example, the command \texttt{C-c C-p} from inside the cell
\begin{verbatim}
\beginmaxima
diff(x*sin(x),x);
\endmaxima
\end{verbatim}
\noindent
will result in
\begin{verbatim}
\beginmaximasession
diff(x*sin(x),x);
\endmaximasession
\end{verbatim}
\noindent
If a standard cell is an initialization cell or a package part, its
type cannot be changed.


\section{Miscellaneous}

\noindent
Some \mx{} commands can be used even outside of cells.  The command 
\texttt{C-c C-u l} send the current line to a
\mx{} process, comment out the current line, and insert the \mx{}
output in the current buffer.  The command 
\texttt{C-c C-u L} will do the same, but
return the result in \LaTeX{} form.

The command \texttt{C-c C-h} will provide
information on a prompted for function (like \mx's \texttt{describe}), 
and  \texttt{C-c C-i} will give the \mx{} info manual.

Finally, the \mx{} process can be killed with \texttt{C-c C-k}.

\newpage
\appendix

\section{Installation}

\noindent
The \emx{} package consists of the files \texttt{maxima.el},
\texttt{emaxima.el},\\
\texttt{maxima-symbols.el},
\texttt{maxima-font-lock.el}, \texttt{emaxima.sty} and \texttt{emaxima.lisp}.
To install, place the \texttt{.el} files, as well as
\texttt{emaxima.lisp}\footnote{If Emacs cannot find
  \texttt{emaxima.lisp}, then the \TeX{} output functions will not
  work, any attempts to get \TeX{} output will only result in standard
  output.} 
somewhere in the load path for Emacs.
Finally, if you want to run \LaTeX{} on the resulting document, put
\texttt{emaxima.sty} somewhere in the \TeX{} inputs path.  If you use
pdflatex, you'll also need \texttt{pdfcolmk.sty}.

To make sure that \texttt{emaxima.el} is loaded when necessary, the line
\begin{verbatim}
(autoload 'emaxima-mode "emaxima" "EMaxima" t)
\end{verbatim}
\noindent
can be inserted into your \texttt{.emacs} file.  Then typing
\texttt{M-x emaxima-mode} will start \emx{} mode.  The command 
\texttt{M-x emaxima-mark-file-as-emaxima} will put the line
\begin{verbatim}
%-*-EMaxima-*-
\end{verbatim}
\noindent
at the beginning of the file, if it isn't there already, and will ensure
that the next time the file is opened, it will be in \texttt{emaxima-mode}.  
This can be done automatically everytime a file is put in
\texttt{emaxima-mode} by putting the line
\begin{verbatim}
(add-hook 'emaxima-mode-hook 'emaxima-mark-file-as-emaxima)
\end{verbatim}
\noindent
somewhere in your \texttt{.emacs} file.

\section{Customizing EMaxima}

\noindent
There are a few (very few) things that you can do to customize \emx{}.  

By default, \emx{} is an extension of AUC\TeX{} mode.  This can be
changed by changing the variable \texttt{emaxima-use-tex}.  The possible
values are \texttt{'auctex}, \texttt{'tex} and \texttt{nil}.  Setting
\texttt{emaxima-use-tex} (the default) to \texttt{'auctex} will make \emx{}
an extension of AUC\TeX{}, setting it to \texttt{'tex} will make \emx{} an
extension of Emacs's default \TeX{} mode, and setting
\texttt{emaxima-use-tex} to \texttt{nil} will make \emx{} an extension of
text-mode.  So, for example, putting 
\begin{verbatim}
(setq emaxima-use-tex nil)
\end{verbatim}
\noindent
in your \texttt{.emacs} file will make \emx{} default to an extension of
text mode. 

Whether or not the dots (\dots{}) abbreviation is allowed in cell
references is controlled by the elisp variable
\texttt{emaxima-abbreviations-allowed}, which is set to \texttt{t} by
default.  Setting this to \texttt{nil} will disallow the abbreviations,
but will speed up package assembly.

The \LaTeX{}ed output can also be configured in a couple of ways.
The lines that appear around cells when the document is \TeX{}ed can be
turned off with the command (in the \LaTeX{} document)
\begin{verbatim}
\maximalinesfalse
\end{verbatim}
\noindent
They can be turned back on with the command
\begin{verbatim}
\maximalinestrue
\end{verbatim}
\noindent

The fonts used to display the \mx{} input and output in a cell are by
default \texttt{cmtt10}.  They can be changed, seperately, by changing the
\TeX{} values of \verb+\maximainputfont+ and \verb+\maximaoutputfont+.
So, for example, to use \texttt{cmtt12} as the input font, use the command
\begin{verbatim}
\font\maximainputfont = cmtt12
\end{verbatim}
\noindent
The spacing in the cells can be controlled by changing the \TeX{}
variables \verb+\maximainputbaselineskip+ and
\verb+\maximaoutputbaselineskip+, and so to increase the space between
the lines of the output, the command
\begin{verbatim}
\maximaoutputbaselineskip = 14pt
\end{verbatim}
\noindent
could be used.
The amount of space that appears before a cell can be changed by changing
the value of \verb+\premaximaspace+ (by default, 0pt), and that after
a cell can be changed by changing the value of \verb+\postmaximaspace+
(by default, 1.5 ex).
 
Session cells can be configured similarly.  
Lines can be placed around a \mx{} session with the command
\begin{verbatim}
\maximasessionlinestrue
\end{verbatim}
\noindent
and they can be turned back off with
\begin{verbatim}
\maximasessionlinesfalse
\end{verbatim}
\noindent
The font can be changed by changing the value of
\verb+\maximasessionfont+.  The color of the prompts when the session
is in \TeX{} form is controlled by \\
\verb+\maximapromptcolor+, by
default red, the colors of the input lines and output lines are
controlled by \verb+\maximainputcolor+ and \verb+\maximaoutputcolor+,
respectively. So the command
\begin{verbatim}
\def\maximainputcolor{green}
\end{verbatim}
\noindent
would make the input in a \TeX{}ed session green.  
The session can be \TeX{}ed without the colors by using the command
\verb+\maximasessionnocolor+.
The baselineskip is
set by \verb+\maximasessionbaselineskip+ for normal session cells, and
by \verb+\maixmatexsessionbaselineskip+ for \TeX{} sessions.  The
amount of space that appears before a session cell can be changed by
changing the value of \verb+\premaximasessionspace+ (by default, 0pt),
and that after a cell can be changed by changing the value of
\verb+\postmaximasessionspace+ (by default, 1.5 ex).

\section{\mx{} mode}

\noindent
\mx{} mode is a major mode for writing \mx{} code.
For moving around in the code, \mx{} mode provides the following
motion commands: 

\begin{tabular}{p{\firstcol}p{\secondcol}}
\texttt{M-C-a} & Move to the beginning of the form.\\
\texttt{M-C-e} & Move to the end of the form.\\
\texttt{M-C-b} & Move to the beginning of the sexp.\\
\texttt{M-C-f} & Move to the end of the sexp.
\end{tabular}

\noindent
and the following miscellaneous commands:

\begin{tabular}{p{\firstcol}p{\secondcol}}
\texttt{C-c C-f} & Mark the current form.\\
\texttt{C-c )} & Check the current region for balanced parentheses.\\
\texttt{C-c C-)} & Check the current form for balanced parentheses.
\end{tabular}

\noindent
\mx{} mode has the following completion command:

\begin{tabular}{p{\firstcol}p{\secondcol}}
\texttt{M-TAB} & Complete the Maxima symbol as much as possible, providing
     a completion buffer if there is more than one possible
     completion.  (If the variable
     \texttt{maxima-use-dynamic-complete} is non-nil, then
     \texttt{M-TAB} will cycle through possible completions.
\end{tabular}

Portions of the buffer can be sent to a Maxima process.  (If a process is 
not running, one will be started.)

\begin{tabular}{p{\firstcol}p{\secondcol}}
\texttt{C-cC-r} & Send the region to Maxima.\\
\texttt{C-cC-b} & Send the buffer to Maxima.\\
\texttt{C-cC-c} & Send the line to Maxima.\\
\texttt{C-cC-e} & Send the form to Maxima.\\
\texttt{C-RET} & Send the smallest set of lines which contains
the cursor and contains no incomplete forms, and go to the next form.\\
\texttt{M-RET} &  As above, but with the region instead of the current line.\\
\texttt{C-cC-l} & Prompt for and load a Maxima file.
\end{tabular}

\noindent
When something is sent to Maxima, a buffer running an inferior Maxima 
process will appear.  It can also be made to appear by using the command
\texttt{C-c C-p}.
If an argument is given to a command to send information to Maxima,
the region (buffer, line, form) will first be checked to make sure
the parentheses are balanced.
The Maxima process can be killed, after asking for confirmation 
with \texttt{C-cC-k}.  To kill without confirmation, give \texttt{C-cC-k}
an argument.

By default, a newline will be indented to the same level as the 
previous line, with an additional space added for open parentheses.
A tab will add extra spaces, as determined by the value of the 
variable \texttt{maxima-indent-amount}.  By default, this is 2.
The behaviour of newline and indent can be changed by the command 
\texttt{M-x maxima-change-newline-style}.  The possibilities are:
\begin{description}
\item[Basic] A newline will have no indentation, and indentation
               must be added with tabs.
\item[Standard]      As above.
\item[Perhaps smart] Tries to guess an appropriate indentation, based on
               open parentheses, "do" loops, etc.
               A newline will re-indent the current line, then indent
               the new line an appropriate amount.
\end{description}
The default can be set by setting the value of the variable 
\texttt{maxima-newline-style} to either \texttt{'basic}, 
\texttt{'standard} or \texttt{'perhaps-smart}.
In all cases, \texttt{M-x maxima-untab} will remove a level of indentation.

To get help on a Maxima topic, use \texttt{C-c C-d}.
To read the Maxima info manual, use \texttt{C-c C-m}.
To get help with the symbol under point, use \texttt{C-cC-h}.
To get apropos with the symbol under point, use \texttt{C-cC-a}.


\section{Running \mx{} Interactively}

\noindent
To run \mx{} interactively in a buffer, type \texttt{M-x maxima}.
In the \mx{} process buffer,
return will check the line for balanced parentheses, and send line as input.
Control return will send the line as input without checking for balanced
parentheses.  The following commands are also available.

\smallskip

\begin{tabular}{p{\firstcol}p{\secondcol}}
\texttt{M-TAB} & Complete the Maxima symbol as much as possible, providing
     a completion buffer if there is more than one possible
     completion.  (If the variable
     \texttt{maxima-use-dynamic-complete} is non-nil, then
     \texttt{M-TAB} will cycle through possible completions.\\
\texttt{C-M-TAB} & Complete the input line, based on previous input lines.\\
\texttt{C-c C-d} & Get help on a Maxima topic.\\
\texttt{C-c C-m} & Bring up the Maxima info manual.\\
\texttt{C-cC-k} & Kill the process and the buffer, after asking for
  confirmation.  To kill without confirmation, give \texttt{C-cC-k} an
  argument.\\
\texttt{M-p} & Bring the previous input to the current prompt.\\
\texttt{M-n} & Bring the next input to the prompt.\\
\texttt{M-r} & Bring the previous input matching
  a regular expression to the prompt.\\
\texttt{M-s} & Bring the next input matching
  a regular expression to the prompt.
\end{tabular}



\newpage

\section{\emx{} mode commands}

\noindent
\begin{tabular}{p{\firstcol}p{\secondcol}}
\hline
\textbf{Key} & \textbf{Description}\\
\hline
\texttt{C-c C-o} & Create a (standard) cell.\\
\texttt{C-c C-p} & Create a session cell.\\
\texttt{C-c C-n} & Create a noshow cell.\\
\texttt{C-c +} & Go the the next cell.\\
\texttt{C-c -} & Go to the previous cell.\\
\texttt{C-c C-u a} & 
Update all of the cells.  With an argument, don't ask before updating.\\
\texttt{C-c C-u A}
& Update all of the cells in \TeX{} form. With an argument don't ask
before updating.\\
\texttt{C-c C-u t}
& Evaluate all of the initialization cells.\\
\texttt{C-c C-u i}
& Update all of the initialization cells.  With an argument, don't
ask before updating.\\
\texttt{C-c C-u I}
& Update all of the initialization cells in \TeX{} form.  With an
argument, don't ask before updating.\\
\texttt{C-c C-u s}
& Update all of the session cells in \TeX{} form.  With an
argument, don't ask before updating.
\end{tabular}

\smallskip

\noindent
\textbf{Commands only available in cells.}

\smallskip

\noindent
\begin{tabular}{p{\firstcol}p{\secondcol}}
\hline
\textbf{Key} & \textbf{Description}\\
\hline
\texttt{C-c C-v}
%& \texttt{emaxima-send-cell}
& Send the current cell to the \mx{} process.\\
\texttt{C-c C-u c}
%& \texttt{emaxima-update-cell}
& Update the current cell.\\
\texttt{C-c C-u C}
%& \texttt{emaxima-tex-update-cell}
& Update the current cell in \TeX{} form.\\
\texttt{C-c C-d}
%& \texttt{emaxima-delete-output}
& Delete the output from the current cell.\\
\texttt{C-c C-t}
%& \texttt{emaxima-toggle-init}
& Toggle whether or not the current cell is an initialization cell.\\
\texttt{C-c C-x}
%& \texttt{emaxima-package-part}
& Insert a heading for the cell indicating that it's part of a
package. \\
\texttt{C-c @}
%& \texttt{emaxima-assemble}
& Assemble the references contained in the cell.  With an argument,
assemble the package that the cell defines.\\
\texttt{C-c C-\texttt{TAB}}
%& \texttt{emaxima-insert-complete-name}
& Complete a reference within a cell.
\end{tabular}

\smallskip

\noindent
\textbf{Commands only available outside of cells.}

\smallskip

\noindent
\begin{tabular}{p{\firstcol}p{\secondcol}}
\hline
\textbf{Key} & \textbf{Description}\\
\hline
\texttt{C-c C-u l}
%& \texttt{emaxima-replace-line}
& Send the current line to \mx{}, and replace the line with the
\mx{} output.\\
\texttt{C-c C-u L}
%& \texttt{emaxima-replace-line-with-tex}
& Send the current line to \mx{}, and replace the line with the
\mx{} output in \TeX{} form.
\end{tabular}

\newpage

\section{\mx{} mode commands}


\smallskip

\noindent
\textbf{Motion}

\smallskip

\noindent
\begin{tabular}{p{\firstcol}p{\secondcol}}
\hline
\textbf{Key} & \textbf{Description}\\
\hline
\texttt{M-C-a} & Go to the beginning of the form.\\
\texttt{M-C-e} & Go to the end of the form.\\
\texttt{M-C-b} & Go to the beginning of the sexp.\\
\texttt{M-C-f} & Go to the end of the sexp.
\end{tabular}

\smallskip

\noindent
\textbf{Process}

\smallskip

\noindent
\begin{tabular}{p{\firstcol}p{\secondcol}}
\hline
\textbf{Key} & \textbf{Description}\\
\hline
\texttt{C-cC-p} & Start a \mx{} process.\\
\texttt{C-cC-r} & Send the region to the \mx{} process.\\
\texttt{C-cC-b} & Send the buffer to the \mx{} process.\\
\texttt{C-cC-c} & Send the line to the \mx{} process.\\
\texttt{C-cC-e} & Send the form to the \mx{} process.\\
\texttt{C-cC-k} & Kill the \mx{} process.\\
\texttt{C-cC-p} & Display the \mx{} buffer.
\end{tabular}

\smallskip

\noindent
\textbf{Completion}

\smallskip

\noindent
\begin{tabular}{p{\firstcol}p{\secondcol}}
\hline
\textbf{Key} & \textbf{Description}\\
\hline
\texttt{M-TAB} & Complete the \mx{} symbol.\\
\end{tabular}

\smallskip

\noindent
\textbf{Comments}

\smallskip

\noindent
\begin{tabular}{p{\firstcol}p{\secondcol}}
\hline
\textbf{Key} & \textbf{Description}\\
\hline
\texttt{C-c ;} & Comment the region.\\
\texttt{C-c :} & Uncomment the region.\\
\texttt{M-;} & Insert a short comment.\\
\texttt{C-c *} & Insert a comment environment.
\end{tabular}


\smallskip

\noindent
\textbf{Indentation}

\smallskip

\noindent
\begin{tabular}{p{\firstcol}p{\secondcol}}
\hline
\textbf{Key} & \textbf{Description}\\
\hline
\texttt{TAB} & Indent line.\\
\texttt{M-C-q} & Indent form.
\end{tabular}


\smallskip

\noindent
\textbf{\mx{} help}

\smallskip

\noindent
\begin{tabular}{p{\firstcol}p{\secondcol}}
\hline
\textbf{Key} & \textbf{Description}\\
\hline
\texttt{C-c C-d}
%& \texttt{maxima-help}
& Get help on a (prompted for) subject.\\
\texttt{C-c C-m}
%& \texttt{maxima-apropos}
& Read the manual.\\
\texttt{C-cC-h} & Get help with the symbol under point.\\
\texttt{C-cC-a} & Get apropos with the symbol under point.
\end{tabular}

\smallskip

\noindent
\textbf{Miscellaneous}

\smallskip

\noindent
\begin{tabular}{p{\firstcol}p{\secondcol}}
\hline
\textbf{Key} & \textbf{Description}\\
\hline
\texttt{M-h} & Mark the form.\\
\texttt{C-c)} & Check the region for balanced parentheses.\\
\texttt{C-c C-)} & Check the form for balanced parentheses.
\end{tabular}

\section{AUC\TeX{} commands}

\smallskip

\noindent
\textbf{Inserting commands}

\smallskip

\noindent
\begin{tabular}{p{\firstcol}p{\secondcol}}
\hline
\textbf{Key} & \textbf{Description}\\
\hline
\texttt{C-c C-e}
& Insert an environment.\\
\texttt{C-c C-s}
& Insert a section.\\
\texttt{C-c ]}
& Close an environment.\\
\texttt{C-c C-j}
& Insert an item into a list.\\
\texttt{"}
& Smart quote.\\
\texttt{\$}
& Smart dollar sign.\\
\texttt{C-c @}
& Insert double brace.\\
\texttt{C-c C-m}
& Insert \TeX{} macro.\\
\texttt{M-TAB}
& Complete \TeX{} macro.\\
\end{tabular}

\smallskip

\noindent
\textbf{Formatting}

\smallskip

\noindent
\begin{tabular}{p{\firstcol}p{\secondcol}}
\hline
\textbf{Key} & \textbf{Description}\\
\hline
\texttt{C-c C-q C-r}
& Format region.\\
\texttt{C-c C-q C-s}
& Format section.\\
\texttt{C-c C-q C-e}
& Format environment.\\
\texttt{C-c .}
& Mark an environment.\\
\texttt{C-c *}
& Mark a section.
\end{tabular}

\smallskip

\noindent
\textbf{Commenting}

\smallskip

\noindent
\begin{tabular}{p{\firstcol}p{\secondcol}}
\hline
\textbf{Key} & \textbf{Description}\\
\hline
\texttt{C-c ;}
& Comment a region.\\
\texttt{C-u C-c ;}
& Uncomment a region.\\
\texttt{C-c \%}
& Comment a paragraph.\\
\texttt{C-u C-c \%}
& Uncomment a paragraph.
\end{tabular}

\smallskip

\noindent
\textbf{Font selection}

\smallskip

\noindent
\begin{tabular}{p{\firstcol}p{\secondcol}}
\hline
\textbf{Key} & \textbf{Description}\\
\hline
\texttt{C-c C-f C-b}
& Bold.\\
\texttt{C-c C-f C-i}
& Italics.\\
\texttt{C-c C-f C-r}
& Roman.\\
\texttt{C-c C-f C-e}
& Emphasized.\\
\texttt{C-c C-f C-t}
& Typewriter.\\
\texttt{C-c C-f C-s}
& Slanted.\\
\texttt{C-c C-f C-d}
& Delete font.\\
\texttt{C-u C-c C-f}
& Change font.
\end{tabular}

\newpage

\noindent
\textbf{Running \TeX{}}

\smallskip

\noindent
(Commands: \texttt{TeX}, \texttt{TeX Interactive}, \texttt{LaTeX},
\texttt{LaTeX Interactive}, \texttt{SliTeX}, \texttt{View},
\texttt{Print}, \texttt{BibTeX}, \texttt{Index}, \texttt{Check},
\texttt{File}, \texttt{Spell}.)

\smallskip

\noindent
\begin{tabular}{p{\firstcol}p{\secondcol}}
\hline
\textbf{Key} & \textbf{Description}\\
\hline
\texttt{C-c C-c}
& Run a command on the master file.\\
\texttt{C-c C-r}
& Run a command on the current region.\\
\texttt{C-c C-b}
& Run a command on the buffer.\\
\texttt{C-c `}
& Go to the next error.\\
\texttt{C-c C-k}
& Kill the \TeX{} process.\\
\texttt{C-c C-l}
& Center the output buffer.\\
\texttt{C-c C-\^{}}
& Switch to the master file.\\
\texttt{C-c C-w}
& Toggle debug of overful boxes.\\
\end{tabular}
\end{document}